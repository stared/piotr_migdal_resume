%%%%%%%%%%%%%%%%%
% This is an sample CV template created using altacv.cls
% (v1.6.5, 3 Nov 2022) written by LianTze Lim (liantze@gmail.com), based on the
% CV created by BusinessInsider at http://www.businessinsider.my/a-sample-resume-for-marissa-mayer-2016-7/?r=US&IR=T
%
%% It may be distributed and/or modified under the
%% conditions of the LaTeX Project Public License, either version 1.3
%% of this license or (at your option) any later version.
%% The latest version of this license is in
%%    http://www.latex-project.org/lppl.txt
%% and version 1.3 or later is part of all distributions of LaTeX
%% version 2003/12/01 or later.
%%%%%%%%%%%%%%%%

%% Use the "normalphoto" option if you want a normal photo instead of cropped to a circle
% \documentclass[10pt,a4paper,normalphoto]{altacv}

\documentclass[10pt,a4paper,ragged2e,withhyper]{altacv}

%% AltaCV uses the fontawesome5 package.
%% See http://texdoc.net/pkg/fontawesome5 for full list of symbols.

% Change the page layout if you need to
\geometry{left=1.25cm,right=1.25cm,top=1.5cm,bottom=1.5cm,columnsep=1.2cm}

% The paracol package lets you typeset columns of text in parallel
\usepackage{paracol}


% Change the font if you want to, depending on whether
% you're using pdflatex or xelatex/lualatex
\ifxetexorluatex
  % If using xelatex or lualatex:
  \setmainfont{Lato}
\else
  % If using pdflatex:
  \usepackage[default]{lato}
\fi

% Change the colours if you want to
\definecolor{VividPurple}{HTML}{3E0097}
\definecolor{SlateGrey}{HTML}{2E2E2E}
\definecolor{LightGrey}{HTML}{666666}
% \colorlet{name}{black}
% \colorlet{tagline}{PastelRed}
\colorlet{heading}{VividPurple}
\colorlet{headingrule}{VividPurple}
% \colorlet{subheading}{PastelRed}
\colorlet{accent}{VividPurple}
\colorlet{emphasis}{SlateGrey}
\colorlet{body}{LightGrey}

% Change some fonts, if necessary
% \renewcommand{\namefont}{\Huge\rmfamily\bfseries}
% \renewcommand{\personalinfofont}{\footnotesize}
% \renewcommand{\cvsectionfont}{\LARGE\rmfamily\bfseries}
% \renewcommand{\cvsubsectionfont}{\large\bfseries}

% Change the bullets for itemize and rating marker
% for \cvskill if you want to
\renewcommand{\itemmarker}{{\small\textbullet}}
\renewcommand{\ratingmarker}{\faCircle}

%% Use (and optionally edit if necessary) this .tex if you
%% want to use an author-year reference style like APA(6)
%% for your publication list
% % When using APA6 if you need more author names to be listed
% because you're e.g. the 12th author, add apamaxprtauth=12
\usepackage[backend=biber,style=apa6,sorting=ydnt]{biblatex}
\defbibheading{pubtype}{\cvsubsection{#1}}
\renewcommand{\bibsetup}{\vspace*{-\baselineskip}}
\AtEveryBibitem{%
  \makebox[\bibhang][l]{\itemmarker}%
  \iffieldundef{doi}{}{\clearfield{url}}%
}
\setlength{\bibitemsep}{0.25\baselineskip}
\setlength{\bibhang}{1.25em}


%% Use (and optionally edit if necessary) this .tex if you
%% want an originally numerical reference style like IEEE
%% for your publication list
\usepackage[backend=biber,style=ieee,sorting=ydnt]{biblatex}
%% For removing numbering entirely when using a numeric style
\setlength{\bibhang}{1.25em}
\DeclareFieldFormat{labelnumberwidth}{\makebox[\bibhang][l]{\itemmarker}}
\setlength{\biblabelsep}{0pt}
\defbibheading{pubtype}{\cvsubsection{#1}}
\renewcommand{\bibsetup}{\vspace*{-\baselineskip}}
\AtEveryBibitem{%
  \iffieldundef{doi}{}{\clearfield{url}}%
}


%% sample.bib contains your publications
\addbibresource{sample.bib}

\begin{document}
\name{Piotr Migdał, PhD}
\tagline{Deep Learning Specialist, Data Science Tech Lead}
% Cropped to square from https://en.wikipedia.org/wiki/Marissa_Mayer#/media/File:Marissa_Mayer_May_2014_(cropped).jpg, CC-BY 2.0
%% You can add multiple photos on the left or right
\photoR{2.5cm}{piotr migdal face direct smiling 2022 by cytacka.jpg}
% \photoL{2cm}{Yacht_High,Suitcase_High}
\personalinfo{%
  % Not all of these are required!
  % You can add your own with \printinfo{symbol}{detail}
  \email{pmigdal@gmail.com}
%   \phone{000-00-0000}
  \location{remote \& Warsaw, Poland}\\
  \homepage{p.migdal.pl}
  \twitter{pmigdal}
  \linkedin{piotrmigdal}
  \github{github.com/stared} % I'm just making this up though.
%   \orcid{0000-0000-0000-0000} % Obviously making this up too.
  %% You can add your own arbitrary detail with
  %% \printinfo{symbol}{detail}[optional hyperlink prefix]
  % \printinfo{\faPaw}{Hey ho!}
  %% Or you can declare your own field with
  %% \NewInfoFiled{fieldname}{symbol}[optional hyperlink prefix] and use it:
  % \NewInfoField{gitlab}{\faGitlab}[https://gitlab.com/]
  % \gitlab{your_id}
	%%
  %% For services and platforms like Mastodon where there isn't a
  %% straightforward relation between the user ID/nickname and the hyperlink,
  %% you can use \printinfo directly e.g.
  % \printinfo{\faMastodon}{@username@instace}[https://instance.url/@username]
  %% But if you absolutely want to create new dedicated info fields for
  %% such platforms, then use \NewInfoField* with a star:
  % \NewInfoField*{mastodon}{\faMastodon}
  %% then you can use \mastodon, with TWO arguments where the 2nd argument is
  %% the full hyperlink.
  % \mastodon{@username@instance}{https://instance.url/@username}
}

\makecvheader

%% Depending on your tastes, you may want to make fonts of itemize environments slightly smaller
\AtBeginEnvironment{itemize}{\small}

%% Set the left/right column width ratio to 6:4.
\columnratio{0.6}

% Start a 2-column paracol. Both the left and right columns will automatically
% break across pages if things get too long.
\begin{paracol}{2}

\cvsection{Experience}

\cvevent{co-founder \& CTO}{Quantum Flytrap}{2020 -- 2022}{Singapore / Warsaw, Poland}
\begin{itemize}
\item A startup making quantum easy for business -- by developing a quantum computing IDE.
\item Delivered a web-based quantum computing interface to Pasqal, a leading quantum hardware manufacturer.
\item Key tech: TypeScript, Vue, Rust, quantum computing.
\end{itemize}

\divider

\cvevent{AI researcher}{ECC Games}{2020 -- 2021}{Warsaw, Poland}
\begin{itemize}
\item Deep learning models for simulating racing car physics \& unsupervised image segmentation.
\item Key tech: Python, PyTorch.
\end{itemize}

\divider

\cvevent{machine and deep learning consulting}{independent}{2015 -- 2019}{US, Europe}
\begin{itemize}
\item Selected clients: deepsense.ai, Intel, Nielsen, Samsung Research, and BCG.
\item Key tech: Python, Jupyter Notebook, Scikit Learn, PyTorch, Keras.
\end{itemize}

\divider

\cvevent{data science and data viz freelancing}{independent}{2014 -- 2015}{remote \& Warsaw, Poland}
\begin{itemize}
\item Selected clients: Data4Cure, Startup Compass, Laboratorium EE.
\item Key tech: Python, Pandas, NumPy, SciPy, JavaScript, D3.js.
\end{itemize}

\divider

\cvevent{PhD in quantum optics theory}{ICFO -- The Institute of Photonic Sciences}{2014 -- 2015}{Castelldefels, Spain}
\begin{itemize}
\item Advisors: Maciej Lewenstein, Javier Rodriguez-Laguna.
\item Reviewers: Seth Lloyd (MIT), Karol Życzkowski (UJ).
\item Key tech: Mathematica, LaTeX, Python.
\end{itemize}

%\cvsection{A Day of My Life}

% Adapted from @Jake's answer from http://tex.stackexchange.com/a/82729/226
% \wheelchart{outer radius}{inner radius}{
% comma-separated list of value/text width/color/detail}
% Some ad-hoc tweaking to adjust the labels so that they don't overlap
%\hspace*{-1em}  %% quick hack to move the wheelchart a bit left
%\wheelchart{1.5cm}{0.5cm}{%
%  10/13em/accent!30/Sleeping \& dreaming about work,
%   25/9em/accent!60/Public resolving issues with Yahoo!\ investors,
%   5/11em/accent!10/\footnotesize\\[1ex]New York \& San Francisco Ballet Jawbone board member,
%   20/11em/accent!40/Spending time with family,
%   5/8em/accent!20/\footnotesize Business development for Yahoo!\ after the Verizon acquisition,
%   30/9em/accent/Showing Yahoo!\ \mbox{employees} that their work has meaning,
%   5/8em/accent!20/Baking cupcakes
% }

% use ONLY \newpage if you want to force a page break for
% ONLY the currentc column
% \newpage

% \cvsection{Publications}

% %% Specify your last name(s) and first name(s) as given in the .bib to automatically bold your own name in the publications list.
% %% One caveat: You need to write \bibnamedelima where there's a space in your name for this to work properly; or write \bibnamedelimi if you use initials in the .bib
% %% You can specify multiple names, especially if you have changed your name or if you need to highlight multiple authors.
% \mynames{Lim/Lian\bibnamedelima Tze,
%   Wong/Lian\bibnamedelima Tze,
%   Lim/Tracy,
%   Lim/L.\bibnamedelimi T.}
% %% MAKE SURE THERE IS NO SPACE AFTER THE FINAL NAME IN YOUR \mynames LIST

% \nocite{*}

% \printbibliography[heading=pubtype,title={\printinfo{\faBook}{Books}},type=book]

% \divider

% \printbibliography[heading=pubtype,title={\printinfo{\faFile*[regular]}{Journal Articles}}, type=article]

% \divider

% \printbibliography[heading=pubtype,title={\printinfo{\faUsers}{Conference Proceedings}},type=inproceedings]

%% Switch to the right column. This will now automatically move to the second
%% page if the content is too long.
\switchcolumn

\cvsection{Life Philosophy}
\begin{quote}
``You live as long as you learn.''
\end{quote}

\cvsection{Most Proud of}

\cvachievement{\faChartLine}{livelossplot}{Developed a popular Python library for visualizing deep learning model training, with over 1.5M installations.}

\divider

\cvachievement{\faFemale}{Data Science PL}{Created Data Science PL, the largest data science community in Poland with over 20k members.}

\divider

\cvachievement{\faTrophy}{RSI \& IPhO}{Alumnus of the renowned MIT Research Science Institute and recipient of the bronze medal in the International Physics Olympiad.}

\divider

\cvachievement{\faHeartbeat}{Inspiration \& impact}{15 of my projects and posts have been featured on the Hacker News front page.}

\divider

\cvachievement{\faExpandArrows*}{Broad interests}{My research publications span a wide range of disciplines from quantum physics to mathematical psychology, educational games, and data visualization.}

%\cvsection{Strengths}
%
%\cvtag{Hard-working (18/24)}
%\cvtag{Persuasive}\\
%\cvtag{Motivator \& Leader}
%
%\divider\smallskip
%
%\cvtag{UX}
%\cvtag{Mobile Devices \& Applications}
%\cvtag{Product Management \& Marketing}

\cvsection{Languages}


\cvtag{Polish}
\cvtag{English}

%\cvskill{English}{5}
% \divider

%\cvskill{Spanish}{4}
% \divider

%\cvskill{German}{3.5} %% supports X.5 values.


\cvsection{Education}

\cvevent{PhD\ in Quantum Optics}{ICFO -- The Institute of Photonic Sciences}{2011 -- 2014}{}

\divider

\cvevent{MSc\ in Physics}{University of Warsaw}{2005 -- 2011}{}

\divider

\cvevent{BSc\ in Mathematics}{University of Warsaw}{2005 -- 2009}{}

%\newpage

% \cvsection{Referees}

% % \cvref{name}{email}{mailing address}
% \cvref{Prof.\ Alpha Beta}{Institute}{a.beta@university.edu}
% {Address Line 1\\Address line 2}

% \divider

% \cvref{Prof.\ Gamma Delta}{Institute}{g.delta@university.edu}
% {Address Line 1\\Address line 2}

\end{paracol}

\end{document}
